\title{Lab_1}
\author{Pavel Zhukouski}
\date{September 2021}

\documentclass[a4paper,14pt,usenames,dvipsnames]{extreport}
\usepackage[left=20mm, top=20mm, right=10mm, bottom=10mm, nohead, nofoot]{geometry}
\pagestyle{empty}
\usepackage[english,russian]{babel}
\usepackage{cmap}
% AMS пакеты для отображения и форматирования математических формул.
\usepackage{amssymb,amsthm,amsmath,amscd,setspace}

\newcommand{\mathsym}[1]{{}}
\newcommand{\unicode}[1]{{}}

\newcounter{mathematicapage}

% Пакет для вставки отбражений
\usepackage{graphicx}
\usepackage{float}
\usepackage[T1,T2A]{fontenc}
\usepackage[utf8]{inputenc}
\usepackage{tikz}

\usepackage{csquotes}

\usepackage{kbordermatrix}

%\renewcommand{\rmdefault}{ftm} % Times New Roman
\usetikzlibrary{arrows.meta}
\usepackage{fancyref}

\usepackage{algorithm}
\usepackage{algpseudocode}
\floatname{algorithm}{Алгоритм}

\usepackage{caption2}[2008/03/29]
\renewcommand{\captionlabeldelim}{.~}

\begin{document}

На рисунке \ref{pic:3type} изображен исходный граф №3  

\begin{figure}[H]
\centering
\begin{tikzpicture}[scale = 0.9]

\def \n {7}
\def \radius {3cm}
\def \margin {8} % margin in angles, depends on the radius

\foreach \s in {1,...,\n}
{
  \node[draw, circle, minimum size=1cm] at ({360/\n * (\s + 2)}:\radius) (\s){$\s$};
}

\draw [-{Latex[length=2mm]}, line width=0.5mm] (1) to (7);
\draw [-{Latex[length=2mm]}, line width=0.5mm] (2) to (3);
\draw [-{Latex[length=2mm]}, line width=0.5mm] (2) to (6);
\draw [-{Latex[length=2mm]}, line width=0.5mm] (3) to (1);
\draw [-{Latex[length=2mm]}, line width=0.5mm] (3) to (6);
\draw [-{Latex[length=2mm]}, line width=0.5mm] (4) to (2);
\draw [-{Latex[length=2mm]}, line width=0.5mm] (4) to (7);
\draw [-{Latex[length=2mm]}, line width=0.5mm] (5) to (2);
\draw [-{Latex[length=2mm]}, line width=0.5mm] (5) to (3);
\draw [-{Latex[length=2mm]}, line width=0.5mm] (6) to (1);
\draw [-{Latex[length=2mm]}, line width=0.5mm] (6) to (4);
\draw [-{Latex[length=2mm]}, line width=0.5mm] (7) to (5);

\end{tikzpicture}
\caption{Исходный граф №3}\label{pic:3type}
\end{figure}

На рисунке \ref{pic:3typeostov} выбрано произвольное остовное дерево графа №3.

\begin{figure}[H]
\centering
\begin{tikzpicture}[scale = 0.9]

\def \n {7}
\def \radius {3cm}
\def \margin {8} % margin in angles, depends on the radius

\foreach \s in {1,...,\n}
{
  \node[draw, circle, minimum size=1cm] at ({360/\n * (\s + 2)}:\radius) (\s){$\s$};
}

\draw [-{Latex[length=2mm]}, line width=0.5mm] (2) to (3);
\draw [-{Latex[length=2mm]}, line width=0.5mm] (2) to (6);
\draw [-{Latex[length=2mm]}, line width=0.5mm] (3) to (1);
\draw [-{Latex[length=2mm]}, line width=0.5mm] (4) to (2);
\draw [-{Latex[length=2mm]}, line width=0.5mm] (5) to (2);
\draw [-{Latex[length=2mm]}, line width=0.5mm] (7) to (5);

\end{tikzpicture}
\caption{Произвольное остовное дерево графа №3}\label{pic:3typeostov}
\end{figure}

\subsection*{Система фундаментальных циклов}

Теперь, для всех фундаментальных циклов, порождаемых данным деревом, найдём характеристические векторы.

Всего в графе существует 12 - 7 + 1 = 6 фундаментальных (базисных) циклов, т.к. мы имеем всего 12 дуг и 7 вершин в исходном графе.

Построим множество $\left\{\delta^{k}(\tau, \rho),(\tau, \rho)^{k} \in U^{k} \backslash U_{T}^{k}\right\}$ характеристических векторов относительно выбранного покрывающего дерева.

Соответствующая матрица базисных циклов будет иметь следующий вид:
\begin{equation}  \label{mat:3}
F = \kbordermatrix{\mbox{}
 	  & (1, 7) & (2, 3) & (2, 6) & (3, 1) & (3, 6) & (4, 2) & (4, 7) & (5, 2) & (5, 3) & (6, 1) & (6, 4) & (7, 5) \\
C(1, 7) & 1 & 1 & 0 & 1 & 0 & 0 & 0 & 1 & 0 & 0 & 0 & 1 \\
C(3, 6) & 0 & 1 & -1 & 0 & 1 & 0 & 0 & 0 & 0 & 0 & 0 & 0 \\
C(4, 7) & 0 & 0 & 0 & 0 & 0 & -1 & 1 & 1 & 0 & 0 & 0 & 1 \\
C(5, 3) & 0 & -1 & 0 & 0 & 0 & 0 & 0 & -1 & 1 & 0 & 0 & 0 \\
C(6, 1) & 0 & -1 & 1 & -1 & 0 & 0 & 0 & 0 & 0 & 1 & 0 & 0 \\
C(6, 4) & 0 & 0 & 1 & 0 & 0 & 1 & 0 & 0 & 0 & 0 & 1 & 0
}.
\end{equation}

Вышеуказанная таблица может быть использована для определения базисных векторов (т.е. характеристических векторов базисных циклов) графа №3.

Пусть $C = \{(1,7), (7, 5), (5, 3), (3, 1)\}$ --- некоторый цикл в нашем графе №3. Тогда он может быть представлен в виде линейной комбинации базисных векторов. 
\begin{table}[H]
\renewcommand{\arraystretch}{1.3}
\caption{Характеристические векторы базисных циклов относительно $U_{T}$ и характеристический вектор цикла $C$ }
\label{tab:u3}
\begin{center}
\begin{tabular}{|c|c|c|c|c|c|c|c|}
\hline 
$(i, j)$&$\delta_{i j}(1,7)$&$\delta_{i j}(3,6)$&$\delta_{i j}(4,7)$&$\delta_{i j}(5,3)$&$\delta_{i j}(6,1)$&$\delta_{i j}(6,4)$&$\delta_{i j}(C)$\\
\hline $(1, 7)$ & 1 & 0 & 0 & 0 & 0 & 0 & 1\\
\hline $(2, 3)$ & 1 & 1 & 0 & -1 & -1 & 0 & 0\\
\hline $(2, 6)$ & 0 & -1 & 0 & 0 & 1 & 1 & 0\\
\hline $(3, 1)$ & 1 & 0 & 0 & 0 & -1 & 0 & 1\\
\hline $(3, 6)$ & 0 & 1 & 0 & 0 & 0 & 0 & 0\\
\hline $(4, 2)$ & 0 & 0 & -1 & 0 & 0 & 1 & 0\\
\hline $(4, 7)$ & 0 & 0 & 1 & 0 & 0 & 0 & 0\\
\hline $(5, 2)$ & 1 & 0 & 1 & -1 & 0 & 0 & 0\\
\hline $(5, 3)$ & 0 & 0 & 0 & 1 & 0 & 0 & 1\\
\hline $(6, 1)$ & 0 & 0 & 0 & 0 & 1 & 0 & 0\\
\hline $(6, 4)$ & 0 & 0 & 0 & 0 & 0 & 1 & 0\\
\hline $(7, 5)$ & 1 & 0 & 0 & 0 & 0 & 0 & 1\\
\hline
\end{tabular}
\end{center}
\end{table}
$\delta(C) = \delta_{1,7}(C) \delta(1,7)  +  \delta_{3,6}(C) \delta(3,6) + \delta_{4,7}(C) \delta(4,7)  +  \delta_{5,3}(C) \delta(5,3) + 
\delta_{6,1}(C) \delta(6,1)  +$  \\ $+ \delta_{6,4}(C) \delta(6,4) =
\begin{pmatrix}
1 \\ 
1 \\ 
0 \\ 
1 \\ 
0 \\ 
0 \\
0 \\ 
1 \\ 
0 \\ 
0 \\ 
0 \\ 
1
\end{pmatrix} +
\begin{pmatrix}
0 \\ 
-1 \\ 
0 \\ 
0 \\ 
0 \\ 
0 \\
0 \\ 
-1 \\ 
1 \\ 
0 \\ 
0 \\ 
0
\end{pmatrix} = \begin{pmatrix}
1 \\ 
0 \\ 
0 \\ 
1 \\ 
0 \\ 
0 \\
0 \\ 
0 \\ 
1 \\ 
0 \\ 
0 \\ 
1
\end{pmatrix}.$

\subsection*{Фундаментальные или базисные разрезы}

Построим характеристические векторы базисных разрезов. Характеристический вектор произвольного разреза $CC(I')$ может быть представлен в виде линейной комбинации базисных разрезов, где $I' = \{1, 5, 6, 7\}$:
\begin{gather*}
CC^+(I') = \{(5,2), (5,3), (6,4)\},\\
CC^-(I') = \{(2,6), (3,1), (3, 6), (4, 7)\},\\
CC(I') = \{(5,2), (5,3), (6,4), -(2,6), -(3,1), -(3, 6), -(4, 7)\}.
\end{gather*}
\begin{table}[H]
\renewcommand{\arraystretch}{1.3}
\caption{Характеристические векторы относительно $U_{T}$}
\label{tab:u3}
\begin{center}
\begin{tabular}{|c|c|c|c|c|c|c|c|}
\hline $(i, j)$ & $\tilde{\delta}_{i j}(2,3)$ & $\tilde{\delta}_{i j}(2,6)$ & $\tilde{\delta}_{i j}(3,1)$ & $\tilde{\delta}_{i j}(4,2)$ & $\tilde{\delta}_{i j}(5,2)$ & $\tilde{\delta}_{i j}(7,5)$ & $\tilde{\delta}_{i j}(CC(I'))$\\
\hline $(1,7)$ & -1 & 0 & -1 & 0 & -1 & -1 & 0 \\
\hline $(2,3)$ & 1 & 0 & 0 & 0 & 0 & 0 & 0 \\
\hline $(2,6)$ & 0 & 1 & 0 & 0 & 0 & 0 & -1 \\
\hline $(3,1)$ & 0 & 0 & 1 & 0 & 0 & 0 & -1 \\
\hline $(3,6)$ & -1 & 1 & 0 & 0 & 0 & 0 & -1 \\
\hline $(4,2)$ & 0 & 0 & 0 & 1 & 0 & 0 & 0 \\
\hline $(4,7)$ & 0 & 0 & 0 & 1 & -1 & -1 & -1 \\
\hline $(5,2)$ & 0 & 0 & 0 & 0 & 1 & 0 & 1 \\
\hline $(5,3)$ & 1 & 0 & 0 & 0 & 1 & 0 & 1 \\
\hline $(6,1)$ & 1 & -1 & 1 & 0 & 0 & 0 & 0 \\
\hline $(6,4)$ & 0 & -1 & 0 & -1 & 0 & 0 & 1 \\
\hline $(7,5)$ & 0 & 0 & 0 & 0 & 0 & 1 & 0 \\
\hline
\end{tabular}
\end{center}
\end{table}

Тогда вектор разреза:

\begin{gather*}
\tilde{\delta}(CC(I')) = 
\tilde{\delta}_{2,3}(CC(I')) \tilde{\delta}(2,3) + \tilde{\delta}_{2,6}(CC(I')) \tilde{\delta}(2,6) + \\ + 
\tilde{\delta}_{3,1}(CC(I')) \tilde{\delta}(3,1) + \tilde{\delta}_{4,2}(CC(I')) \tilde{\delta}(4,2) + \\ +
\tilde{\delta}_{5,2}(CC(I')) \tilde{\delta}(5,2) +
\tilde{\delta}_{7,5}(CC(I')) \tilde{\delta}(7,5) = \\
= -\begin{pmatrix}
0 \\ 
0 \\ 
1 \\ 
0 \\ 
1 \\ 
0 \\
0 \\
0 \\
0 \\
-1 \\
-1 \\
0
\end{pmatrix} -
\begin{pmatrix}
-1 \\ 
0 \\ 
0 \\ 
1 \\ 
0 \\ 
0 \\
0 \\
0 \\
0 \\
1 \\
0 \\
0
\end{pmatrix} +
\begin{pmatrix}
-1 \\ 
0 \\ 
0 \\ 
0 \\ 
0 \\ 
0 \\
-1 \\
1 \\
1 \\
0 \\
0 \\
0
\end{pmatrix} = \begin{pmatrix}
0 \\ 
0 \\ 
-1 \\ 
-1 \\ 
-1 \\ 
0 \\
-1 \\
1 \\
1 \\
0 \\
1 \\
0
\end{pmatrix}.
\end{gather*}

\subsection*{Поток в сети}
В качестве источника возьмём вершину 1, а в качестве стока возьмём вершину 7. Тогда математическая модель потока будет иметь вид:

\begin{gather*}
x_{1,7} - x_{3,1} - x_{6,1} = -1,\\
x_{2,3} + x_{2,6} - x_{4,2} - x_{5,2}= 2,\\
x_{3,1} + x_{3,6} - x_{2,3} - x_{5,3} = 3, \\
x_{4,2} + x_{4,7} - x_{6,4} = 4,\\
x_{5,2} + x_{5,3} - x_{7,5} = 5,\\
x_{6,1} + x_{6,4} - x_{2,6} - x_{3,6} = -6, \\
x_{7,5} - x_{1,7} - x_{4,7} = -7.\\
\end{gather*}

\subsection*{Корневое дерево}

\begin{algorithm} [H]
\caption{Процедура нахождения узлов поддерева с корнем в узле $i$}\label{alg:Example}
\begin{algorithmic}
\State $k \gets \operatorname{depth}[i]$
\State $j \gets \operatorname{thread}[i]$
\While {$\operatorname{depth}[j] > k$}
    \State $j \gets \operatorname{thread}[j]$
\EndWhile
\end{algorithmic}
\end{algorithm}

На рисунке \ref{pic:3typekornder} изображено корневое дерево графа №3 с корнем в узле 1.

\begin{figure}[H]
\centering
\begin{tikzpicture}[scale = 0.9]

\def \n {7}
\def \radius {3cm}
\def \margin {8} % margin in angles, depends on the radius

\foreach \s in {1,...,\n}
{
  \node[draw, circle, minimum size=1cm] at ({360/\n * (\s + 2)}:\radius) (\s){$\s$};
}

\draw [-{Latex[length=2mm]}, line width=0.5mm] (3) to (2);
\draw [-{Latex[length=2mm]}, line width=0.5mm] (2) to (6);
\draw [-{Latex[length=2mm]}, line width=0.5mm] (1) to (3);
\draw [-{Latex[length=2mm]}, line width=0.5mm] (2) to (4);
\draw [-{Latex[length=2mm]}, line width=0.5mm] (2) to (5);
\draw [-{Latex[length=2mm]}, line width=0.5mm] (5) to (7);

\end{tikzpicture}
\caption{Корневое дерево $G_0$ графа №3 с корнем в узле 1}\label{pic:3typekornder}
\end{figure}

\begin{table}[H]
\renewcommand{\arraystretch}{1.3}
\caption{- списковые структуры для дерева $G_0$ }
\label{tab:u3}
\begin{center}
\begin{tabular}{|c|c|c|c|c|c|c|c|}
\hline 
Структура / Узел i & 1 & 2 & 3 & 4 & 5 & 6 & 7\\
\hline $pred = \{pred[i], i=\overline{1,|V_0|}\ \}$ & -1 & 3 & 1 & 2 & 2 & 2 & 5\\
\hline $depth = \{depth[i], i=\overline{1,|V_0|}\ \}$ & 0 & 2 & 1 & 3 & 3 & 3 & 4\\
\hline $thread = \{thread[i], i=\overline{1,|V_0|}\ \}$ & 3 & 4 & 2 & 5 & 7 & 1 & 6\\
\hline
\end{tabular}
\end{center}
\end{table}

\end{document}