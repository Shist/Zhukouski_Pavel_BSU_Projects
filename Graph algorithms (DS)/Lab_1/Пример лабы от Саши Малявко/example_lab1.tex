
\documentclass[a4paper,14pt,usenames,dvipsnames]{extreport}
\usepackage[left=20mm, top=20mm, right=10mm, bottom=10mm, nohead, nofoot]{geometry}
\pagestyle{empty}
\usepackage[english,russian]{babel}
\usepackage{cmap}
% AMS пакеты для отображения и форматирования математических формул.
\usepackage{amssymb,amsthm,amsmath,amscd,setspace}

\newcommand{\mathsym}[1]{{}}
\newcommand{\unicode}[1]{{}}

\newcounter{mathematicapage}

% Пакет для вставки отбражений
\usepackage{graphicx}
\usepackage{float}
\usepackage[T1,T2A]{fontenc}
\usepackage[utf8]{inputenc}
\usepackage{tikz}

\usepackage{csquotes}

\usepackage{kbordermatrix}

%\renewcommand{\rmdefault}{ftm} % Times New Roman
\usetikzlibrary{arrows.meta}
\usepackage{fancyref}

\begin{document}

На рисунке \ref{pic:3type} изображен поток в сети. Множество вершин $I^3 = \{2, 3, 4, 6, 7\}$, входящих в данный поток, выделено пунктирными линиями. На рисунке пунктиром изображены только те дуги, через которые проходит поток --- дуги множества $U^3 = \{ (2, 3)^3, (4, 2)^3, (6, 2)^3, (6, 4)^3, (7,4)^3, (7, 6)^3\}$.  

\begin{figure}[H]
\centering
\begin{tikzpicture}[scale = 0.9]

\def \n {7}
\def \radius {3cm}
\def \margin {8} % margin in angles, depends on the radius

\foreach \s in {1,...,\n}
{
  \node[draw, circle, minimum size=1cm] at ({360/\n * (\s - 1)}:\radius) (\s){$\s$};
}

\foreach \s in {2, 3, 4, 6, 7}
{
  \node[draw, circle, minimum size=1.3cm,  dashed] at ({360/\n * (\s - 1)}:\radius) (\s){$\s$};
}

\draw [-{Latex[length=2mm]}, dashed] (7) to (6);
\draw [-{Latex[length=2mm]}, dashed] (7) to (4);
\draw [-{Latex[length=2mm]}, dashed] (6) to (4);
\draw [-{Latex[length=2mm]}, dashed] (6) to (2);
\draw [-{Latex[length=2mm]}, dashed] (4) to (2);
\draw [-{Latex[length=2mm]}, dashed] (2) to (3);

\end{tikzpicture}
\caption{Третий тип потока}\label{pic:3type}
\end{figure}

\begin{figure}[H]
\centering
\begin{tikzpicture}[scale = 0.9]


\def \n {7}
\def \radius {3cm}
\def \margin {8} % margin in angles, depends on the radius

\foreach \s in {1,...,\n}
{
  \node[draw, circle, minimum size=1cm] at ({360/\n * (\s - 1)}:\radius) (\s){$\s$};
}

\draw [-{Latex[length=4mm]}, dashed, line width=0.5mm] (7) to (6);
\draw [-{Latex[length=4mm]}, dashed, line width=0.5mm] (6) to (4);
%\draw [-{Latex[length=2mm]}, dashed] (6) to (2);
\draw [-{Latex[length=4mm]}, dashed, line width=0.5mm] (4) to (2);
\draw [-{Latex[length=4mm]}, dashed, line width=0.5mm] (2) to (3);

\end{tikzpicture}
\caption{Остовное дерево для третьего типа потока}\label{pic:3typeostov}
\end{figure}

\subsection*{Базисные циклы}

Построим множество $\left\{\delta^{k}(\tau, \rho),(\tau, \rho)^{k} \in U^{k} \backslash U_{T}^{k}\right\}$ характеристических векторов относительно выбранного покрывающего дерева.

Соответствующая матрица базисных циклов будет иметь следующий вид:
\begin{equation}  \label{mat:3}
F_3 = \kbordermatrix{\mbox{}
 	  & (2,3) & (4,2) & (6,2) & (6,4) & (7,4) & (7,6) \\
C^3(6, 2) & 0 & -1 & 1 & -1 & 0 & 0 \\
C^3(7, 4) & 0 & 0 & 0 & -1 & 1 & -1
}.
\end{equation}

Пусть $C^3 = \{(7,6)^3, (6, 2)^3, -(4, 2)^3, -(7, 4)^1\}$ --- некоторый цикл в сети $S^3$.  Тогда он может быть представлен в виде линейной комбинации базисных векторов. 
\begin{table}[H]
\renewcommand{\arraystretch}{1.3}
\caption{Характеристические векторы базисных циклов относительно $U_{T}^{3}$ и характеристический вектор цикла $C^3$ }
\label{tab:u3}
\begin{center}
\begin{tabular}{|c|c|c|c|c|c|c|}
\hline $(i, j)^{3}$ & 
$(2,3)^{3}$&$(4,2)^{3}$&$(6,2)^{3}$&$(6,4)^{3}$&$(7,4)^{3}$&$(7,6)^{3}$ \\
\hline $\delta_{i j}^{k}(\tau, \rho)=\delta_{i j}^{3}(6,2)$ 
& $0$ & $-1$ & $1$ & $-1$ & $0$ & $0$ \\
\hline $\delta_{i j}^{k}(\tau, \rho)=\delta_{i j}^{3}(7,4)$ 
& $0$ & $0$ & $0$ & $-1$ & $1$ & $-1$ \\
\hline
$\delta_{i j}(C^3)$ 
& $0$ & $-1$ & $1$ & $0$ & $-1$ & $1$ \\
\hline
\end{tabular}
\end{center}
\end{table}
$$\delta(C^3) = \delta_{62}(C^3) \delta^{3}(6,2) + \delta_{74}(C^3) \delta^{3}(7,4) = \begin{pmatrix}
0 \\ 
-1 \\ 
1 \\ 
-1 \\ 
0 \\ 
0
\end{pmatrix} -
\begin{pmatrix}
0 \\ 
0 \\ 
0 \\ 
-1 \\ 
1 \\ 
-1
\end{pmatrix} = \begin{pmatrix}
0 \\ 
-1 \\ 
1 \\ 
0 \\ 
-1 \\ 
1
\end{pmatrix}.$$

\subsection*{Базисные разрезы}

Построим характеристические векторы базисных разрезов, а также найдем характеристический вектор разреза   $CC(I_3')$, где $I_3' = \{2, 3, 7\}$: $CC^+(I_3') = \{(7, 4), (7, 6)\}$, $CC^-(I_3') = \{(6,2), (4,2)\}$, $CC(I_3') = \{(7, 4), (7, 6), -(6,2), -(4,2)\}$.
\begin{table}[H]
\renewcommand{\arraystretch}{1.3}
\caption{Характеристические векторы относительно $U_{T}^{3}$}
\label{tab:u3}
\begin{center}
\begin{tabular}{|c|c|c|c|c|c|c|}
\hline $(i, j)^{3}$ & 
$(2,3)^{3}$&$(4,2)^{3}$&$(6,2)^{3}$&$(6,4)^{3}$&$(7,4)^{3}$&$(7,6)^{3}$ \\ \hline 
$\tilde{\delta}_{i j}^{3}(2,3)$ 
& $1$ & $0$ & $0$ & $0$ & $0$ & $0$ \\ \hline 
$\tilde{\delta}_{i j}^{3}(4,2)$ 
& $0$ & $1$ & $1$ & $0$ & $0$ & $0$ \\ \hline
$\tilde{\delta}_{i j}^{3}(6,4)$ 
& $0$ & $0$ & $1$ & $1$ & $1$ & $0$ \\ \hline
$\tilde{\delta}_{i j}^{3}(7,6)$ 
& $0$ & $0$ & $0$ & $0$ & $1$ & $1$ \\ \hline
$\tilde{\delta}_{i j}(CC(I_3'))$
& $0$ & $-1$ & $-1$ & $0$ & $1$ & $1$ \\
\hline
\end{tabular}
\end{center}
\end{table}

\begin{gather*}
\tilde{\delta}(CC(I_3')) = 
\tilde{\delta}_{23}(CC(I_3')) \tilde{\delta}^{3}(2,3) +\tilde{\delta}_{42}(CC(I_3')) \tilde{\delta}^{3}(4,2) + \\ + 
\tilde{\delta}_{64}(CC(I_3')) \tilde{\delta}^{3}(6,4) + \tilde{\delta}_{76}(CC(I_3')) \tilde{\delta}^{3}(7,6)  = \\ = -\begin{pmatrix}
0 \\ 
1 \\ 
1 \\ 
0 \\ 
0 \\ 
0
\end{pmatrix} +
\begin{pmatrix}
0 \\ 
0 \\ 
0 \\ 
0 \\ 
1 \\ 
1
\end{pmatrix} = \begin{pmatrix}
0 \\ 
-1 \\ 
-1 \\ 
0 \\ 
1 \\ 
1
\end{pmatrix}.
\end{gather*}

\subsection*{Поток в сети}
Математическая модель потока будет иметь вид:

\begin{gather*}
c_{74}x_{74} + c_{76}x_{76} = \nu,\\
c_{64}x_{64} + c_{62}x_{62} - c_{76}x_{76} = 0,\\
c_{42}x_{42} - c_{64}x_{64} - c_{74}x_{74} = 0, \\
c_{23}x_{23} - c_{42}x_{42} - c_{62}x_{62} = 0,\\
-c_{23}x{23} = -\nu,\\
\end{gather*}

\end{document}